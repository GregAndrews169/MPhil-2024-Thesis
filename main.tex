\documentclass[a4paper,11pt,fleqn]{report}

\usepackage{acronym}
\usepackage{amsmath,amssymb,amsfonts}
\usepackage{booktabs}
\usepackage[dvipsnames]{xcolor}
\usepackage{lipsum}  
\usepackage[margin=30mm]{geometry}
\usepackage{graphicx}
	\graphicspath{ {../../graphics/} }
\usepackage{float}
\usepackage{hyperref}
	\hypersetup{
		colorlinks=true,
		linkcolor=blue,
		filecolor=blue,
		urlcolor=blue,
		citecolor=blue
	}
\usepackage[sort&compress]{natbib}
	\bibliographystyle{apalike}

\usepackage{soul}
\usepackage{url}
\usepackage{wasysym}

\newcommand{\rjg}[1]{{\color{Red}{~(rjg: #1)}}}

%%%%%%%%%%%%%%%%%%%%%%%%%%%%%%%%%%%%%%%%%%%%%%%%%%%%%%%%%%%%%%%%%%%%%%%%%%%%%%%%%%%%%%%%%%%%%%%%%%%%

\begin{document}
	
\thispagestyle{empty}

% Play around with the spacing here, but this is a better title page for your final report...

\begin{center}
	{\huge Exploring alternative credit assessment methodologies for lower income individuals within Africa}
	\vspace{20mm} \\
	{\Large Gregory Raymond Andrews}
	\vfill
	
	A project report in partial fulfilment of the requirements for the degree \\
	\vspace{10mm}
	{\Large \textsc{Master of Philosophy (Financial Technology)}} \\
	\vfill
	%
	in the \\
	\vspace{20mm}
	%
	{\Large \textsc{Faculty of Commerce}}\\
	%
	\vspace{10mm}
	{\Large\textsc{University of Cape Town}} \\
	%
	\vfill
	%
	\today
\end{center}



\pagenumbering{roman}
\tableofcontents
\listoffigures
\listoftables



\chapter{Introduction}
\pagenumbering{arabic}
\setcounter{page}{1}
\acresetall



\section{Background}

In the dynamic world of global finance, the significance of credit scoring in assessing the creditworthiness of individuals and Small and Medium-sized Enterprises (SMEs) is paramount. Traditional credit scoring methods rely extensively on historical financial data derived from formal banking interactions and established credit relationships. However, this conventional methodology faces substantial challenges, particularly in regions like Africa, where a significant segment of the population and SMEs operate predominantly outside the formal banking system. This thesis aims not only to explore alternative credit scoring methodologies but also to develop an appropriate model that could revolutionize financial inclusion for these groups in Africa.
\\\\
Credit access is a crucial factor in economic growth and personal financial stability. In essence, credit scoring is a predictive tool used to evaluate the likelihood of a borrower repaying a loan. This system influences individuals' access to finance and the terms of credit, such as interest rates. For SMEs, the implications are far-reaching; credit access can be the difference between business growth and stagnation, or even survival and collapse.
\\\\
The financial landscape of many African countries is marked by a high incidence of informal economies. A significant proportion of individuals and SMEs either lack a formal credit history or are underbanked. Traditional credit scoring models, dependent on formal credit histories, income proofs, and banking relationships, often marginalize these segments. This not only restricts their credit access but also impedes their full participation in the economy.
\\\\
Technological advancements, particularly the widespread adoption of mobile money services across Africa, open new possibilities for credit scoring methodologies. These technological strides offer an opportunity to utilize alternative data sources, like mobile money transaction histories, utility bill payments, and social media activities, for creditworthiness assessment. This alternative data can provide a more comprehensive view of an individual's or SME's financial behavior, especially when traditional credit data is lacking.
\\\\
Thus, this thesis embarks on an exploration of these alternative credit scoring methods. Beyond analyzing the mechanics of these new models, the research will focus on their potential to bridge the financial inclusion gap. It will investigate the effectiveness of alternative data in predicting creditworthiness, the challenges in implementing these models, and their potential impact on individuals and SMEs in Africa.
\\\\
A significant ambition of this thesis is to develop a suitable alternative credit scoring model, tailored to the African context. This involves not just theoretical exploration but practical model formulation, considering the nuances of the African financial ecosystem.
\\\\
By undertaking this comprehensive study, the thesis aims to contribute significantly to the discourse on financial inclusion. This is a topic of increasing relevance amid global economic changes and the digital revolution. Exploring the synergy between technology, finance, and social development, this research aspires to illuminate innovative paths for integrating a substantial segment of the African populace into the formal financial system. The ramifications of such an endeavor could be transformative, not only for Africa but for similar markets globally, setting the stage for a more inclusive financial future.

\section{Problem description}

The prevailing methodologies of credit scoring in Africa are characterized by systemic inadequacies that disproportionately affect lower-income individuals and Small and Medium-sized Enterprises (SMEs). This results in a significant exclusion of these groups from accessing essential financial services. The crux of the problem lies in three interrelated issues:

\begin{enumerate}
  \item \textbf{Inadequate Representation in Traditional Credit Systems:} A large segment of Africa's population and SMEs operates outside the formal banking sector, leading to a scarcity of traditional credit histories. Conventional credit scoring models, which rely heavily on these histories, fail to accommodate the financial realities of these individuals and businesses. This misalignment results in a substantial portion of the population being unjustly categorized as ‘uncreditworthy’, not due to their actual financial behavior or potential, but due to their lack of presence in traditional financial systems.
  
  \item \textbf{Limitations of Current Credit Scoring Models:} The existing models in use are predominantly tailored to Western financial behaviors and norms, which do not align well with the African context. These models tend to have inherent biases and limitations in predictive accuracy, especially for those with minimal or no formal credit history. Consequently, this leads to a distorted view of creditworthiness that often excludes otherwise viable borrowers based on outdated or irrelevant criteria.
  
  \item \textbf{Underutilization of Alternative Data in Credit Assessment:} Despite the rapid advancement in digital financial services across Africa and the proliferation of rich alternative data sources such as mobile money transactions, utility bill payments, and digital footprints, these are not effectively utilized in existing credit scoring processes. This represents a significant missed opportunity in harnessing technology for financial inclusion.
\end{enumerate}

This thesis identifies a crucial research gap in the exploration and development of alternative credit scoring models that are specifically tailored to the unique financial landscape and needs of Africa. The challenge lies in creating a model that not only incorporates alternative data sources but does so in a manner that is accurate, fair, and predictive of actual credit risk.
\\\\
The overarching goal of this research is to develop an innovative credit scoring model that transcends the limitations of traditional systems. Such a model would not only enable a more comprehensive and realistic assessment of creditworthiness for a broader section of the African population but also pave the way for enhanced financial inclusion, contributing to economic empowerment and growth. By tackling these issues, the research aims to offer a transformative solution to a problem that is both a barrier to financial accessibility and a bottleneck in the economic development of the continent.

\section{Research design}

The research design for this thesis is systematically structured to address the identified challenges in credit scoring within the African context, employing a blend of qualitative and quantitative research methods.

\subsection{Phase 1: Exploratory Research and Framework Development}
\begin{itemize}
    \item \textbf{Literature Review:} Conduct an in-depth review of existing credit scoring models, focusing on their application and limitations in both African and global contexts.
    \item \textbf{Conceptual Framework Development:} Develop a framework for an alternative credit scoring model, emphasizing the integration of alternative data sources.
\end{itemize}

\subsection{Phase 2: Data Collection and Analysis}
\begin{itemize}
    \item \textbf{Qualitative Data Gathering:} Perform interviews and focus groups with financial experts, SME owners, and target demographic representatives in various African countries.
    \item \textbf{Quantitative Data Gathering:} Collect empirical data including traditional financial records and alternative data sources like mobile money transactions and utility payments.
\end{itemize}

\subsection{Phase 3: Model Development and Initial Testing}
\begin{itemize}
    \item \textbf{Initial Model Formulation:} Develop a preliminary alternative credit scoring model based on collected data.
    \item \textbf{Prototype Testing:} Test the model using a controlled dataset for initial evaluation of accuracy and feasibility.
\end{itemize}

\subsection{Phase 4: Model Refinement and Validation}
\begin{itemize}
    \item \textbf{Iterative Model Refinement:} Refine the model based on initial testing results, adjusting parameters and incorporating new data as necessary.
    \item \textbf{Comprehensive Validation:} Extensively validate the refined model to ensure its reliability and accuracy across diverse scenarios in the African context.
\end{itemize}

\subsection{Phase 6: Finalization and Recommendations}
\begin{itemize}
    \item \textbf{Final Model Development:} Finalize the model based on comprehensive validation outcomes.
    \item \textbf{Recommendations and Future Directions:} Offer detailed recommendations for the application of the model in the African financial sector and suggest future research directions.
\end{itemize}

\textbf{Conclusion}
This research design focuses on developing an alternative credit scoring model that is robust, accurate, and tailored to the unique financial environment of Africa. The goal is to enhance financial inclusion by addressing the current limitations in credit access for lower-income individuals and SMEs in the region.

\section{Research methodology}



\subsection{Critical Review}
The initial phase involves a critical review of existing literature and models. This comprehensive analysis serves to understand current credit scoring systems and their applicability, especially in the African context.

\begin{itemize}
    \item Reviewing existing credit scoring models and methodologies globally, with a focus on their limitations and biases.
    \item Analyzing the socio-economic factors influencing credit scoring in African markets, including the prevalence of informal economic activities.
    \item Identifying gaps in current methodologies that fail to address the financial realities of lower-income groups and SMEs in Africa.
\end{itemize}

\subsection{The Foundation of Knowledge}
This phase establishes the theoretical and conceptual framework upon which the research will build. It involves synthesizing the information gathered during the critical review into a coherent foundation.

\begin{itemize}
    \item Developing a theoretical framework that encompasses alternative data sources and methodologies suitable for the African context.
    \item Formulating hypotheses about how alternative data (e.g., mobile money transactions, utility payments) can be integrated into credit scoring models.
    \item Designing the research approach for empirical investigation, including data collection strategies and analysis methods.
\end{itemize}

\subsection{Investigation and Implementation}
The final phase focuses on the practical application of the research findings. It involves the collection of data, development of the alternative credit scoring model, and analysis of its efficacy.

\begin{itemize}
    \item \textbf{Data Collection:}
    \begin{itemize}
        \item Collecting both qualitative and quantitative data, including interviews with financial experts and target demographics, as well as empirical financial data.
        \item Ensuring ethical data collection practices, adhering to data protection and privacy regulations.
    \end{itemize}
    
    \item \textbf{Model Development:}
    \begin{itemize}
        \item Developing the credit scoring model based on the theoretical framework and collected data.
        \item Incorporating machine learning techniques and statistical analysis to refine the model for accuracy and inclusiveness.
    \end{itemize}

    \item \textbf{Analysis and Validation:}
    \begin{itemize}
        \item Testing the model using various data subsets to evaluate its predictive power and fairness.
        \item Refining the model based on testing outcomes and stakeholder feedback.
    \end{itemize}
\end{itemize}


\section{Document structure}

This thesis begins with an introductory chapter that sets the stage for a deep dive into the development of an alternative credit scoring model tailored for the African market. Chapter 1 lays the groundwork, outlining the research topic, problem statement, and overall significance of the study. It brings into focus the current landscape of credit scoring methods, highlighting their limitations, especially when applied within the unique financial environment of Africa. This chapter is crucial in establishing the need for an innovative, more inclusive credit scoring system.
\\\\
In Chapter 2, the thesis transitions into an exhaustive literature review. This chapter is dedicated to exploring the existing body of work on credit scoring models. It delves into the nuances of their applications and the inherent limitations they possess, with a keen focus on their applicability and shortcomings in the context of African economies. This review also extends to the global use of alternative data in credit scoring, drawing parallels to the African scenario, thus setting the stage for the proposed alternative model.
\\\\
The methodology adopted for this research is detailed in Chapter 3. It outlines a multi-faceted approach, encompassing the critical review of existing models, establishing a solid theoretical foundation for the new model, and detailing the strategies for empirical investigation and implementation. This chapter is pivotal as it outlines the blueprint for the development of the alternative credit scoring model, including the approaches for data collection, model development, and validation.
\\\\
Chapter 4 is where the thesis takes a practical turn, illustrating the development process of the new credit scoring model. This chapter not only details the incorporation of alternative data sources but also discusses the methodologies used to refine and validate the model. It represents the core of the thesis, bringing to life the theoretical concepts discussed in previous chapters.
\\\\
The penultimate chapter, Chapter 5, is dedicated to analyzing the results derived from applying the new model. This chapter is critical as it discusses the model's effectiveness, the challenges encountered, and the potential impact it holds. It is here that the thesis offers detailed recommendations for the model’s implementation and future enhancements, grounding the theoretical work in practical applicability.
\\\\
Concluding the thesis, Chapter 6 encapsulates the journey of the research. It summarizes the key findings and the significant contributions the study makes to the field of credit scoring and financial inclusion in Africa. The chapter also opens doors to future research opportunities in this domain, suggesting paths for further exploration and development.
\\\\
This structured approach ensures a comprehensive and seamless progression through the thesis, from understanding the underlying issues and reviewing existing literature, to developing and analyzing an innovative solution, culminating in actionable conclusions and recommendations.



\chapter{Literature review}



\section{Overview of Credit Scoring Systems}
This section provides an overview of the concept of credit scoring, discussing its importance and evolution. It will explore the traditional mechanisms of credit scoring, laying a foundation for understanding the current practices in the financial sector.

\section{Traditional Credit Scoring Models}
This section delves into traditional credit scoring models. It discusses the methodologies and algorithms commonly used, such as FICO scores, and their application in different financial contexts. The focus will be on the reliance on formal financial histories and the implications for borrowers.

\section{Limitations of Traditional Credit Scoring}
\subsection{Challenges in Developing Economies}
This subsection examines the specific challenges faced by traditional credit scoring methods in developing economies, particularly in the context of Africa.
\subsection{Impact on Low-Income Individuals and SMEs}
Here, the focus is on how traditional credit scoring methods may fail to accurately represent low-income individuals and SMEs, often leading to their exclusion from formal financial systems.

\section{Alternative Data in Credit Scoring}
This section introduces the concept of using alternative data in credit scoring. It explores various types of non-traditional data, such as mobile money transactions, utility payments, and other digital footprints, discussing their potential in enhancing credit scoring models.

\section{Developing Credit Scoring Models for African Markets}
\subsection{Challenges and Opportunities}
This subsection discusses the unique challenges and opportunities in developing credit scoring models specifically for African markets.
\subsection{Case Studies and Examples}
Here, case studies and examples of successful alternative credit scoring models in similar markets or contexts will be presented.

\section{Technological Advancements in Credit Scoring}
\subsection{Machine Learning and AI in Credit Scoring}
This subsection explores the role of machine learning and AI technologies in developing advanced credit scoring models.
\subsection{Impact of Digitalization on Financial Services}
Here, the focus is on how digitalization in financial services is reshaping credit scoring, with a particular emphasis on African markets.

\section{Regulatory and Ethical Considerations}
This section addresses the regulatory and ethical considerations in developing and implementing new credit scoring models, particularly concerning data privacy and consumer protection.

\section{Synthesis and Research Gap}
\subsection{Summary of Current Knowledge}
This subsection synthesizes the information and insights gathered from the literature.
\subsection{Identification of Research Gap}
Here, the specific research gap that the thesis aims to address will be articulated, setting the stage for the subsequent chapters.

\textbf{Conclusion}
This chapter concludes by summarizing the key findings from the literature review and their implications for the development of an alternative credit scoring model tailored for the African context.

\chapter{Preliminary design approach}



\section{Development}



\subsection{Scenario set-up}



\subsection{Data collection methodology} 


 
\end{document}
